Planting seeds is often a labor-intensive and time-consuming task, particularly for individual
households and small-scale farmers who rely on manual methods involving digging and
carefully placing seeds into the ground. Recognizing the need for a more efficient and userfriendly
solution,
SeedStick
has
been
developed
as
a
semi-automated,
handheld
seed
planter

designed
to
streamline
the
planting
process.
SeedStick
aims
to
simplify
the
task
by

incorporating
three
core
functions
which
are
effective
seed
storage
and
dispensing,
precise
seed
placement,
and optimal planting depth control. The concept generation process for
SeedStick was deeply informed by the principles of applied physics in industrial design, a core
focus of the course. This interdisciplinary approach facilitated an understanding of the
mechanics of seed movement, the forces involved in soil penetration, and the ergonomics of
handheld tools. By applying these principles, a design was created that maximizes efficiency
and ease of use. The final design of SeedStick includes a rotating drum for reliable seed
storage, a syringe tip dispenser for accurate seed placement, an adjustable depth control plate
to ensure the proper planting depth, and an extendable grip for enhanced user comfort. To
validate the design, comprehensive concept testing will be conducted using a mixed-method
approach to gather feedback from target users, which include gardeners and small-scale
farmers. Furthermore, securing patent protection for SeedStick is essential to maintain
exclusivity, attract investment, and gain a competitive advantage in the market. By addressing
the key challenges of traditional seed planting methods and leveraging the principles of
applied physics in industrial design, SeedStick promises to make the planting process more
accessible, efficient, and effective for its users. This development process highlights the
practical application of studies in applied physics and industrial design, ensuring that SeedStick
is not only innovative but also grounded in solid engineering principles. 